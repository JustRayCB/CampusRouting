\documentclass[journal, a4paper]{IEEEtran}
\usepackage[T1]{fontenc}       % Encodage le plus étendu
\usepackage[utf8]{inputenc}    % Source Unicode en UTF-8

\usepackage[cyr]{aeguill}
\usepackage[francais]{babel} % Pour la redaction du document en francais


% some very useful LaTeX packages include:

%\usepackage{cite}      % Written by Donald Arseneau
                        % V1.6 and later of IEEEtran pre-defines the format
                        % of the cite.sty package \cite{} output to follow
                        % that of IEEE. Loading the cite package will
                        % result in citation numbers being automatically
                        % sorted and properly "ranged". i.e.,
                        % [1], [9], [2], [7], [5], [6]
                        % (without using cite.sty)
                        % will become:
                        % [1], [2], [5]--[7], [9] (using cite.sty)
                        % cite.sty's \cite will automatically add leading
                        % space, if needed. Use cite.sty's noadjust option
                        % (cite.sty V3.8 and later) if you want to turn this
                        % off. cite.sty is already installed on most LaTeX
                        % systems. The latest version can be obtained at:
                        % http://www.ctan.org/tex-archive/macros/latex/contrib/supported/cite/

\usepackage{graphicx}   % Written by David Carlisle and Sebastian Rahtz
                        % Required if you want graphics, photos, etc.
                        % graphicx.sty is already installed on most LaTeX
                        % systems. The latest version and documentation can
                        % be obtained at:
                        % http://www.ctan.org/tex-archive/macros/latex/required/graphics/
                        % Another good source of documentation is "Using
                        % Imported Graphics in LaTeX2e" by Keith Reckdahl
                        % which can be found as esplatex.ps and epslatex.pdf
                        % at: http://www.ctan.org/tex-archive/info/

%\usepackage{psfrag}    % Written by Craig Barratt, Michael C. Grant,
                        % and David Carlisle
                        % This package allows you to substitute LaTeX
                        % commands for text in imported EPS graphic files.
                        % In this way, LaTeX symbols can be placed into
                        % graphics that have been generated by other
                        % applications. You must use latex->dvips->ps2pdf
                        % workflow (not direct pdf output from pdflatex) if
                        % you wish to use this capability because it works
                        % via some PostScript tricks. Alternatively, the
                        % graphics could be processed as separate files via
                        % psfrag and dvips, then converted to PDF for
                        % inclusion in the main file which uses pdflatex.
                        % Docs are in "The PSfrag System" by Michael C. Grant
                        % and David Carlisle. There is also some information
                        % about using psfrag in "Using Imported Graphics in
                        % LaTeX2e" by Keith Reckdahl which documents the
                        % graphicx package (see above). The psfrag package
                        % and documentation can be obtained at:
                        % http://www.ctan.org/tex-archive/macros/latex/contrib/supported/psfrag/

%\usepackage{subfigure} % Written by Steven Douglas Cochran
                        % This package makes it easy to put subfigures
                        % in your figures. i.e., "figure 1a and 1b"
                        % Docs are in "Using Imported Graphics in LaTeX2e"
                        % by Keith Reckdahl which also documents the graphicx
                        % package (see above). subfigure.sty is already
                        % installed on most LaTeX systems. The latest version
                        % and documentation can be obtained at:
                        % http://www.ctan.org/tex-archive/macros/latex/contrib/supported/subfigure/

\usepackage{url}        % Written by Donald Arseneau
                        % Provides better support for handling and breaking
                        % URLs. url.sty is already installed on most LaTeX
                        % systems. The latest version can be obtained at:
                        % http://www.ctan.org/tex-archive/macros/latex/contrib/other/misc/
                        % Read the url.sty source comments for usage information.

%\usepackage{stfloats}  % Written by Sigitas Tolusis
                        % Gives LaTeX2e the ability to do double column
                        % floats at the bottom of the page as well as the top.
                        % (e.g., "\begin{figure*}[!b]" is not normally
                        % possible in LaTeX2e). This is an invasive package
                        % which rewrites many portions of the LaTeX2e output
                        % routines. It may not work with other packages that
                        % modify the LaTeX2e output routine and/or with other
                        % versions of LaTeX. The latest version and
                        % documentation can be obtained at:
                        % http://www.ctan.org/tex-archive/macros/latex/contrib/supported/sttools/
                        % Documentation is contained in the stfloats.sty
                        % comments as well as in the presfull.pdf file.
                        % Do not use the stfloats baselinefloat ability as
                        % IEEE does not allow \baselineskip to stretch.
                        % Authors submitting work to the IEEE should note
                        % that IEEE rarely uses double column equations and
                        % that authors should try to avoid such use.
                        % Do not be tempted to use the cuted.sty or
                        % midfloat.sty package (by the same author) as IEEE
                        % does not format its papers in such ways.

\usepackage{amsmath}    % From the American Mathematical Society
                        % A popular package that provides many helpful commands
                        % for dealing with mathematics. Note that the AMSmath
                        % package sets \interdisplaylinepenalty to 10000 thus
                        % preventing page breaks from occurring within multiline
                        % equations. Use:
%\interdisplaylinepenalty=2500
                        % after loading amsmath to restore such page breaks
                        % as IEEEtran.cls normally does. amsmath.sty is already
                        % installed on most LaTeX systems. The latest version
                        % and documentation can be obtained at:
                        % http://www.ctan.org/tex-archive/macros/latex/required/amslatex/math/

\usepackage{lipsum} % Dummy text


% Other popular packages for formatting tables and equations include:

%\usepackage{array}
% Frank Mittelbach's and David Carlisle's array.sty which improves the
% LaTeX2e array and tabular environments to provide better appearances and
% additional user controls. array.sty is already installed on most systems.
% The latest version and documentation can be obtained at:
% http://www.ctan.org/tex-archive/macros/latex/required/tools/

% V1.6 of IEEEtran contains the IEEEeqnarray family of commands that can
% be used to generate multiline equations as well as matrices, tables, etc.

% Also of notable interest:
% Scott Pakin's eqparbox package for creating (automatically sized) equal
% width boxes. Available:
% http://www.ctan.org/tex-archive/macros/latex/contrib/supported/eqparbox/

% *** Do not adjust lengths that control margins, column widths, etc. ***
% *** Do not use packages that alter fonts (such as pslatex).         ***
% There should be no need to do such things with IEEEtran.cls V1.6 and later.


% En-tête et pied de page
%\usepackage{lastpage}
%\usepackage{fancyhdr}
%\pagestyle{fancy}
%\renewcommand{\sectionmark}[1]{\markright{#1}}
%\fancyhead{}
%%\fancyhead[RO,LE]{\slshape\footnotesize\nouppercase{\rightmark}}
%\fancyhead[LO,RE]{\thetitle}
%\fancyfoot{}
%%\fancyfoot[LO,RE]{\footnotesize\texttt{\thefilename}\\ \textit{\now}}
%%\fancyfoot[C]{-~\thepage~/~\pageref{LastPage}~-}
%\fancyfoot[RO,LE]{\raisebox{-2mm}{\includegraphics{structure/barrette-original}}}
%%
%\fancypagestyle{plain}{ %  Première page ----------------------
%  \fancyhead{}
%  \renewcommand{\headrulewidth}{0pt}
%  \fancyheadoffset[R]{15mm}
%  \fancyhead[L]{
%    \raisebox{-7mm}{
%      \parbox{\textwidth}{
%        \includegraphics{structure/barrette-original} \\ \\
%        \fontsize{8pt}{10pt}\selectfont
%        \sffamily\color{Pantone287}
%        FACULTÉ DES SCIENCES       \\
%        DÉPARTEMENT D'INFORMATIQUE   
%      }
%    }
%  }
%  \fancyhead[R]{
%    \raisebox{-10mm}[0pt][0pt]{\includegraphics[width=120mm]{structure/ULB-ligne-gauche}}
%  }
%  \fancyfoot{}
%  %\fancyfoot[L]{\raisebox{0mm}{}\color{Pantone287}\footnotesize\texttt{\thefilename}\\ \textit{\now}}
%  %\fancyfoot[C]{-~\thepage~/~\pageref{LastPage}~-}
%  \fancyfoot[R]{
%    \raisebox{-12pt}{\includegraphics[height=\footskip]{structure/sceau-mini-b-quadri}}
%  }
%} % Fin de première page
% ---------------------------------------------------------------------------


% Your document starts here!
\begin{document}

% Define document title and author
	\title{Template Rapport Scientifique}
	\author{Prenom Nom, Prenom Nom, Prenom Nom
	\thanks{Superviseur: xxx}}
	\markboth{INFO-F308}{}
	\maketitle

% Write abstract here
\begin{abstract}
	Le résumé (80-100 mots) est conçu pour donner au lecteur une vue générale du contenu de l'article. 
\end{abstract}

% Each section begins with a \section{title} command
\section{Introduction}
	% \PARstart{}{} creates a tall first letter for this first paragraph
	\PARstart{C}{ette} section donne une introduction générale du problème scientifique abordé et décrit la structure de l'article. Des questions souvent abordées ici sont :
	\begin{itemize}
	\item Quelles sont les applications du problème abordées ?
	\item Pourquoi la résolution du problème est importante ?
	\end{itemize}	  

\section{Etat de l'art}
	Cette section permet de décrire l'état de l'art concernant la question abordée (c-à-d les meilleures solutions disponibles à présent) et de positionner votre travail par rapport à cet état de l'art.
	Les différents articles que vous avez lus et utilisés doivent être correctement référencés (Exemple: \cite{small},\cite{big}).
	Les informations bibliographiques doivent être encodées dans le fichier \texttt{References.bib} avec la syntaxe indiquée par les exemples.
	Articles publiés sur une revue scientifique / dans les conference proceedings, ainsi que des livres, sont des exemples de bonnes références.
	Par contre, la citation de sources web doit être limitée le plus possible (permis dans le cas de la documentation d'outils informatiques).

% Main Part
\section{Méthodologie}\label{sec:met}
	Cette section doit décrire, en manière détaillée:
	\begin{itemize}
	\item Les hypothèses de base de votre approche
	\item Les fondements mathématiques
	\item La méthode proposée
	\item Les jeux de données utilisés (si nécessaire)
	\item Les instructions nécessaires pour pouvoir reproduire les expériences (par exemple pseudo-code), 
	\end{itemize}
	
	Ici vous pouvez trouver deux exemples de notation mathématique:
	\begin{equation} 
	 f(x)=(x+a)(x+b)
	\end{equation}

	Maxwell's equations:
\begin{align}
        B'&=-\nabla \times E,\\
        E'&=\nabla \times B - 4\pi j,
\end{align}

% Main Part
\section{Résultats}
	Cette section doit contenir les résultats que vous avez obtenu avec la méthodologie décrite dans la section \ref{sec:met}.
	Les résultats devront être présentés de préférence sous forme de tableau (cf. Table~\ref{tab:simParameters}) et/ou du diagramme (cf. Fig.~\ref{fig:tf_plot}), et correctement référencés.
	Les conditions d'expérimentation (c-à-d matériel et logiciels utilisés) devront être ainsi indiquées.
	En plus des résultats mêmes, cette section devra contenir votre propre analyse et discussion de résultats (par exemple comparaison par rapport à une méthode de référence)

% This is how you define a table: the [!hbt] means that LaTeX is forced (by the !) to place the table exactly here (by h), or if that doesnt work because of a pagebreak or so, it tries to place the table to the bottom of the page (by b) or the top (by t).
	\begin{table}[!hbt]
		% Center the table
		\begin{center}
		% Title of the table
		\caption{Simulation Parameters}
		\label{tab:simParameters}
		% Table itself: here we have two columns which are centered and have lines to the left, right and in the middle: |c|c|
		\begin{tabular}{|c|c|}
			% To create a horizontal line, type \hline
			\hline
			% To end a column type &
			% For a linebreak type \\
			Information message length & $k=16000$ bit \\
			\hline
			Radio segment size & $b=160$ bit \\
			\hline
			Rate of component codes & $R_{cc}=1/3$\\
			\hline
			Polynomial of component encoders & $[1 , 33/37 , 25/37]_8$\\
			\hline
		\end{tabular}
		\end{center}
	\end{table}
	
	% This is how you include a eps figure in your document. LaTeX only accepts EPS or TIFF files.
	\begin{figure}[!hbt]
		% Center the figure.
		\begin{center}
		% Include the eps file, scale it such that it's width equals the column width. You can also put width=8cm for example...
		%\includegraphics[width=\columnwidth]{plot_tf}
		% Create a subtitle for the figure.
		\caption{Simulation results on the AWGN channel. Average throughput $k/n$ vs $E_s/N_0$.}
		% Define the label of the figure. It's good to use 'fig:title', so you know that the label belongs to a figure.
		\label{fig:tf_plot}
		\end{center}
	\end{figure}

\section{Conclusion}
Cette section contient un rappel des contributions / de résultats importants de votre article et éventuellement une indication sur les perspectives de recherche future dans le même domaine.


\bibliographystyle{unsrt}
\bibliography{bibliography}

\newpage
		
\appendices
\section{Consignes}
% Main Part
\subsection*{Document}
	% LaTeX takes complete care of your document layout ...
	Le rapport doit être rédigé de préférence en \LaTeX{} en utilisant ce template.
	La longueur du rapport ne devra pas, en tout cas, dépasser les 6 pages.
	Ce rapport doit être \emph{self-contained}, c-à-d il doit pouvoir être lu et compris sans avoir besoin de se documenter ailleurs.



% Your document ends here!
\end{document}